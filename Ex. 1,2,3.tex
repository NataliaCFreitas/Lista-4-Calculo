% ----- Basics Config -----
\documentclass{article}
\usepackage[utf8]{inputenc}
\usepackage[brazil]{babel}
\usepackage[T1]{fontenc}
\usepackage[lmargin=2cm,tmargin=2cm,rmargin=2cm,bmargin=2cm]{geometry}
\usepackage{verbatim}
\usepackage{xcolor}

% ----- Math -----
\usepackage{minted,amsmath,mathtools,amsfonts,polynom}
%% ----- Algorithm -----
\usepackage{algpseudocode}
\usepackage{algorithm}

% ----- Plots & Graphs -----
\usepackage{tikz}
\usepackage{graphicx}

% ----- Tables -----
\usepackage{longtable}
\usepackage{multirow}

% ----- Theorems -----
\newtheorem{exe}{Exemplo}
\newtheorem{ex}{Exercício}
\newtheorem{obs}{Observação}


% ----- Title/author/date -----
\title{\bf Cálculo Diferencial e Integral \\ Lista 4}
\author{ Guilherme Pereira Amorim \\ Julianna Lerner Naslauski \\ Natália Correia Freitas}
\date{Novembro 2022}
% ----- Document -----
\begin{document}
\maketitle
\[\]\
\section {Respostas}
%----------
\begin{ex}
\begin{eqnarray*}
f(x,y)=& x^4\! +y^3\!=4x^3y^3\! & 3y^2\! + x^4\\ g(x,y)=& x\! + y\!= 1\! & 1\\
\end{eqnarray*}
\\
$\left\{
\begin{aligned}
4x^3\!\ y^3\!= \lambda (1) \\ 
3y^2\!\ x^4\!= \lambda (2)\\ 
x\!+ y\!= \lambda (3)\\ 
\end{aligned}\right.$
$\begin{aligned}
\\
$\left\{
\begin{aligned}
4x^3\!\ y^3\!= 3y^2\!\ x^4\\
4x^3\!\ y\!= 3x^4\\ 
\end{aligned}\right.$
$\begin{aligned}
\implies
\hspace{1cm}&y\!=\dfrac{3}{4}\!x\\

$\left
$\begin{eqnarray*}
&\textbf{Vamos substituir a equação 3:}\textbf{}\\
    &x\! + \dfrac{3}{4}\!x - \!1 =0 \\
    &x\! + \dfrac{3}{4}\!x = \!1\\
    &\dfrac{x}{1}\!+\dfrac{3x}{4}\!= \dfrac{4x+3x}{4}\\
    &\dfrac{4x+3x}{4}\!=1\\
    &\dfrac{7x}{4}\!=1\\
    &\dfrac{7x}{4}\!=1\\
    &x=\dfrac{7}{4}\\
\end{eqnarray*}$
$\left
$\begin{eqnarray*}
\textbf{Vamos substituir a equação 1:}\textbf{}\\
    x\! + y\! - 1\! =0 \\
    \dfrac{3}{7}\!+y = \!1\\
    y\!= 1 - \dfrac{3}{7}\implies\dfrac{7-4}{7} =\dfrac{3}{7} = y\\
\end{eqnarray*}$
$\left
$\begin{aligned}
&\textbf{f }\Biggl(\dfrac{4}{7}\!, \dfrac{3}{7}\Biggl) = \Biggl(\dfrac{4}{7}^4 \cdot \dfrac{3}{7}^3 \!\Biggl) = \dfrac{4^4 \cdot 3^3} {7^7}\textbf{}\\
\\
&\textbf{Portanto P = } \Biggl(\dfrac{4}{7}\!, \dfrac{3}{7}\Biggl) \textbf{é a função no p = } \dfrac{4^4 \cdot 3^3} {7^7} \textbf{do ponto crítico maximizado da função. }\textbf{}\\
\end{aligned}$
\end{ex}
%-----
\begin{ex}
\begin{eqnarray*}
f(x,y)=& x^2\! +y^2\!+3xy\! -x\! + y\\
\nabla f(x,y)=& (2x\! + 3y\! -1\!,2y\! +3x\!+1)\\
\nabla f(x_{0},y_{0})=& (0,0)\\
(2x_{0}\! +3y_{0}\! -1\!,2y_{0}\! +3x_{0}\! +1) =&(0,0)\\
\end{eqnarray*}
$\left\{
\begin{aligned}
2x_{0}\! +3y_{0}\! -1\!= 0 (*3) & \implies
\hspace{1cm}&6x_{0}\! +9y_{0}\!-3\!=0\\
3x_{0}\! +2y_{0}\! +1\! =0 (-2) & \implies
\hspace{1cm}&-6x_{0}\! - 4y_{0}\! -2\!=0 
\end{aligned}\right.$
$\begin{aligned}
&&\hspace{1cm}&+9y_{0}\! -3\!=0\\
&&\hspace{1cm}&- 4y_{0}\! -2\!=0\\
&&&\overline{5y_{0}\! -5\!=0} \implies {y_{0}\!=1}
\end{aligned}$
\\
$\begin{aligned}
\\
\textbf{Vamos substituir yo na equação:} \\
    3x_{0}\! +2y_{0}\! +1\! =0  & \implies 3x_{0}\! +2\! +1\! =0\implies
    && x_{0}\!=\dfrac{-3}{3} \! = -1
\end{aligned}$
\vspace{1cm}
\begin{center}
$\begin{aligned}
    H(x,y)= \begin{vmatrix}
    fxx & fxy\\
    fyx & fyy
\end{vmatrix} = 
\begin{vmatrix}
    2 & 3\\
    3 & 2
\end{vmatrix} = 4-9 = -5 \\
\\
\vspace{1cm}
\textbf{- 5 é menor que 0, então (-1,1) é ponto de sela.}\textbf{}
\vspace{1cm}
\end{aligned}$
\end{center}
\vspace{1cm}
\end{ex}
%---------
\begin{ex}
\begin{eqnarray*}
\textbf{1º passo: função objetiva }\textbf{}\\
f(x,y,z)= (x\!- 1)^2\!+(y\!- 1)^2\! + (z\! - 1)^2 \\
f(x,y,z)= x^2\! -2x\!+ 1\! + y^2 -2y\! +1 \!+z^2\! -2z\! + 1\\
f(x,y,z)= x^2\! -2x\!+ y^2 -2y\!+z^2\! -2z\! + 3\\
g(x,y,z)= 3x\! + y\!- z\! -1\ \implies 3x\! + y\!- z\!=1\\
\\
\vspace{1cm}
\textbf{2º passo: MLG }\textbf{}\\
\nabla f(x,y,z)= \lambda \! \cdot\! \nabla \!g(x,y,z)\\
\nabla f(x,y,z)=(2x\!- 2;\!2y\! -2;\!2z\! -2\!)\\
\nabla g(x,y,z)=(3,\!1,\!-1)\\
(2x\!-2;\!2y\! -2;\!2z\! -2\!)=\lambda \!\cdot (3,\!1,\!-1)\\
\vspace{1cm}
\end{eqnarray*}
$$\left\{
\begin{aligned}
2x\! - 2\!= 3 \lambda  & \implies
\hspace{1cm}&x=\dfrac{3\lambda\! + 2\!}{2}\\
2y\! -2\! = \lambda & \implies
\hspace{1cm}&y=\dfrac{\lambda\! + 2\!}{2}\\
2z\! -2\! = -\lambda & \implies
\hspace{1cm}&z=\dfrac{-\lambda\! + 2\!}{2}\\ 
\end{aligned}\right.$$
\\
$$\begin{aligned}
\textbf{3º passo: Substituir }\textbf{} 3x\! + y\!- z\!= 1\\
3\Biggl(\dfrac{3\lambda\! + 2\!}{2}\Biggl)\!+\Biggl(\dfrac{\lambda\! + 2\!}{2}\Biggl)\!-\Biggl(\dfrac{-\lambda\! + 2\!}{2}\Biggl)=1\\
\dfrac{9\lambda\! + 6\!}{2} \!+ \dfrac{\lambda\! + 2\!}{2}\!-\dfrac{\lambda\! + 2\!}{2}\!=1\implies\dfrac{11\lambda\! + 6\!}{2}\!=1\\
11\lambda\!=-4\implies \lambda\!=\dfrac{-4}{11}
\end{aligned}$$
\\
$$\begin{aligned}
\textbf{4º passo: Substituir o }\textbf{} \lambda \textbf{ nos pontos:}\textbf{}\\
x=3\Biggl(\dfrac{(-4/11)}{2}\Biggl)&&\!y=\Biggl(\dfrac{(-4/11)\!+2}{2}\Biggl)&&\!z=-\Biggl(\dfrac{(-4/11)\!+2}{2}\Biggl)\\
\\
&&\textbf{P =  }\textbf{}\Biggl(-\dfrac{6}{11}\!;\dfrac{9}{11}\!;\dfrac{-9}{11}\Biggl)\\
\end{aligned}$$
%-------NÃO MEXER ------
\begin{document}
\end{document}

 
 